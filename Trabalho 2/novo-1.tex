\documentclass{article}

\usepackage[top=2cm, bottom=2cm, left=3cm, right=2cm]{geometry}
\usepackage{setspace}
\usepackage[portuguese]{babel}
\usepackage[utf8]{inputenc}

\begin{document}
\begin{spacing}{1.5}

% capa
\begin{titlepage} %iniciando a "capa"
\begin{center} %centralizar o texto abaixo
{\large Universidade Federal de Santa Catarina}\\[0.2cm] %0,2cm é a distância entre o texto dessa linha e o texto da próxima
{\large Departamento de Engenharia Mec\^anica}\\[0.2cm]
{\large Curso de Gradua\c{c}\~ao em Engenharia Mec\^anica}\\[0.2cm]
{\large  DISCIPLINA EMC 5412 - TRANSFER\^ENCIA DE CALOR E MEC\^ANICA DOS FLUIDOS COMPUTACIONAL}\\[7.6cm]
{\bf \huge Trabalho 2 - Condu\c{c}\~ao n\~ao linear}\\[5.3cm] % o comando \bf deixa o texto entre chaves em negrito. O comando \huge deixa o texto enorme

{\large Aluno(a): Gusttav Bauermann Lang}\\[0.4cm] % o comando \large deixa o texto grande
{\large Professor: Ant\'onio F\'abio Carvalho da Silva}\\[0.4cm]
{\large Matr\'icula: 13200534}
\end{center} %término do comando centralizar

\vspace{\fill}
\begin{center}
{\large Florian\'opolis}\\[0.1cm]
{\large 31 de Mar\c{c}o de 2017}
\end{center}
\end{titlepage} %término da "capa"

\section{Introdu\c{c}\~ao}

A distribuição de temperatura em uma parede pode ser determinada através da resolução da equação do calor. Para meios isotrópicos e em uma situação de regime permanente, a equação do calor em uma parede plana pode ser determinada pela seguinte equação:
\\
$y=3\sin x$

\end{spacing}




this is a short document to illustrate the basic use of \LaTeX


\noindent $y=3\sin x$

$x^{2+\alpha}_{n+1}$

%comentarios

quero mostrar o porcento (\%) e também o \$

\textit{Italic type}
\textrm{Roman}
\textsf{Sans serif}

\textup{Don't \textbf{overuse} types}

como dar \emph{enfase} em uma palavra 

Environments are portions of the document that we want \LaTeX $\;$to treat differently form the main body:

\begin{itemize}
	\item Item 1
	\item Item 2
	\item Item3

\end{itemize}

\begin{enumerate}
	\item primeiro
	\item segundo
	\item terceiro

\end{enumerate}

\begin{center}
Para centralizar texto\\
pular linha sem espaçar o código
\end{center}

Como Fazer tabelas\\
\\
\begin{tabular}{lrc}
Nome & Idade & Nota\\
\hline
Gusttav & 22 & 10\\
Giga & 24 & 10\\
Sergio & 25 & 10
\end{tabular}
\\
O \{lrc\} faz a primeira coluna se alinhar a esquerda, a segundo na direita e a terceira no centro

\begin{center}
\begin{tabular}{|l| |r|c|}
\hline
Nome & Idade & Nota\\
\hline
Gusttav & 22 & 10\\
Giga & 24 & 10\\
Sergio & 25 & 10\\
\hline

\end{tabular}
\end{center}

\begin{table} [h] %esse h faz aparecer o caption da tabela
\begin{center}
\begin{tabular}{|l| |r|r|}
\hline
& \multicolumn{2}{c|}{Numeros}\\
\cline{2-3}
Nome & Idade & Nota\\
\hline
Gusttav & 22 & 10\\
Giga & 24 & 10\\
Sergio & 25 & 10\\
\hline

\end{tabular}
\caption{Tabela nome}\label{tabela}
\end{center}
\end{table}


\begin{verbatim}
mostrar tudo o que eu quiser nesse espaço, como $ % !, sem precisar usar o \$
\end{verbatim}


Espaçamento entre linhas
\bigskip

ou
\medskip

ou
\smallskip

sao as opcoes para espaçamento vertilcal

\vspace{2.2in} para o tamanho exato que queres deixar %pode ser também {2.2cm} pode usar números negativos

Parara utilizar espacamento \hspace{1cm} vertical, ou

pode ser \hspace{\fill} assim
\vspace{\fill}


\begin{center}
\textbf CAPITULO 3 
\end{center}

Alguns símbolos:

$\not<, \not|$

\ldots eh $\not=$ de ... e tbm de $\cdots$\\
$\alpha,   \phi,   \theta $

$\cos$ ou cos? $\rightarrow$
\[
	x = \frac{i+y}{1+2z^2} + \sqrt{5} \;   para espaco 
\]



$S_N = \sum_{j=1}^N a_j  $

\[
	S_N = \sum_{j=1}^N a_j  
\]

\[
	\int_{x=0}^\infty e^{-x^2} dx
\]



\[
\frac{1}{2} < \sqrt[n]{
\left\{
\frac{1\cdot3\cdots(2n-1)}
{2\cdot4\cdots2n}\right\}} <1.
\]


\[
\frac{1}{2} < \sqrt[n]{
\{
\frac{1\cdot3\cdots(2n-1)}
{2\cdot4\cdots2n\}}} <1.
\]




\[
A =
\left[
\begin{array}{ccc}
1& 1 &1\\
x&y&z\\
x^2&y^2&z^2
\end{array}
\right], \; 
\mathbf{u} =
\left[
\begin{array}{c}
x \\ y \\ z
\end{array}
\right]
\]



\newcommand{\Dtn}{\Delta t_n}
Para nao ser tao chato digitar alguns comandos (muito extensos), eh possivel abreviar dessa maneira.\\
$\Dtn := t_{n+1} - t_n$






\end{document}





































